\begin{resumo}
Este trabalho tem como objetivo desenvolver uma plataforma de ensino online utilizando tecnologias descentralizadas. Como metodologia, utilizou-se a análise comparativa de plataformas de ensino já estabelecidas no mercado para guiar o desenvolvimento em torno das principais funcionalidades encontradas neste tipo de software. Ainda, este trabalho visa compreender as limitações do modelo de arquitetura proposto, considerando as vantagens e desvantagens em comparação a um projeto totalmente centralizado.

 \vspace{\onelineskip}
    
 \noindent
 \textbf{Palavras-chaves}: Aprendizado online. P2P. Descentralização. Centralização. IPFS. OrbitDB. Arquitetura cliente-servidor. 
\end{resumo}
