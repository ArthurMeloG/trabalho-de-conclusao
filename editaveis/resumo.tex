\begin{resumo}
Este trabalho tem como objetivo desenvolver uma aplicação de aprendizado online descentralizada a fim de garantir maior privacidade, segurança e autonomia no compartilhamento e consumo de conteúdos educacionais digitais. Como metodologia, utilizou-se a análise comparativa de plataformas de ensino tradicionais para compreender suas limitações e fragilidades, que são consequência do modelo arquitetural centralizado. Os resultados da análise demonstram que a descentralização pode proporcionar vantagens sobre a arquitetura cliente-servidor, proporcionando resistência à censura e maior controle sobre os dados. Com isso, elaborou-se uma proposta descentralizada que utiliza tecnologias como IPFS e OrbitDB, NextJS, NestJS e PostgreSQL. 

 \vspace{\onelineskip}
    
 \noindent
 \textbf{Palavras-chaves}: aprendizado online. P2P. Descentralização. Centralização. IPFS. OrbitDB. Arquitetura cliente-servidor.
\end{resumo}
