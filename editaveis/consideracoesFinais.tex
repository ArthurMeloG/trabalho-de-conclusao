\chapter[Considerações Finais]{Considerações Finais}
\label{cap:consideracoes}

O objetivo principal deste trabalho foi apresentar a proposta de um software de aprendizado \textit{online} descentralizado, cuja arquitetura fosse capaz de representar uma alternativa viável às soluções educacionais baseadas em infraestruturas totalmente centralizadas. A análise de plataformas consolidadas no mercado permitiu a identificação de funcionalidades essenciais para esse tipo de aplicação, servindo de referência e inspiração para o desenvolvimento da solução proposta.

A pesquisa bibliográfica realizada ao longo do projeto possibilitou a compreensão aprofundada do contexto atual da educação \textit{online}, seus desafios, oportunidades e impactos sociais. Esse embasamento foi essencial para guiar as decisões de projeto, promovendo o alinhamento entre os aspectos técnicos e pedagógicos da plataforma. A partir desse entendimento, foi possível adotar metodologias ágeis como o \textit{Extreme Programming} (XP), que contribuíram para tornar o processo de desenvolvimento mais eficiente, colaborativo e adaptável às necessidades identificadas.

Entre os desafios futuros estão a complexidade técnica para a implementação completa da arquitetura descentralizada, a necessidade de maior maturidade no uso das tecnologias envolvidas e a realização de testes com usuários reais. Essa etapa será fundamental para validar as funcionalidades da plataforma, avaliar sua usabilidade e coletar \textit{feedbacks} que orientem melhorias contínuas no sistema.

Por fim, acredita-se que este trabalho contribui de maneira significativa para a ampliação das discussões sobre educação digital descentralizada, propondo uma abordagem alinhada com princípios de liberdade, segurança e autonomia. Ao explorar caminhos alternativos ao modelo centralizado tradicional, este projeto abre espaço para novas possibilidades na forma de ensinar, aprender e compartilhar conhecimento em ambientes digitais.
