\chapter[Considerações Finais]{Considerações Finais}
\label{cap:consideracoes}

O objetivo principal deste trabalho foi apresentar a proposta de um software de aprendizado online descentralizado, em que sua arquitetura fosse capaz de garantir maior autonomia, segurança e privacidade no compartilhamento e consumo de conteúdo educacionais no formato digital. A análise de plataformas tradicionais identificou que o modelo centralizado impõe restrições significativas aos usuários de plataformas que as utilizam, como riscos de censura, controle sobre monetização, mal uso dos dados de compradores e dependência de servidores proprietários. Com isso, o modelo descentralizado apresenta-se como uma oportunidade de driblar tais características, oferencendo mais democratização no processo de educação online.

A pesquisa bibliográfica do presente trabalho permitiu o entendimento do contexto relacionado à educação online, assim como das limitações das presentes soluções. Isso foi fundamental para que a equipe fosse capaz de construir um sistema que contornasse estas características, elaborando uma metodologia para o desenvolvimento e gerenciamento de atividades que fosse capaz de projetar um software robusto e moderno. Também graças as pesquisas, foi possível incluir métodos e práticas (como \textit{Extreme Programming}) para tornar a construção do presente trabalho ainda mais eficaz.

Como perspectivas futuras, a equipe ainda espera diversos desafios, como a grande curva de aprendizado para implementar a arquitetura proposta. Além disso, a realização de testes com usuários reais pode proporcionar novas ideias para melhorar a solução atual, otimizando a usabilidade do sistema.

Por fim, este trabalho contribui para a expansão das pesquisas sobre educação digital descentralizada, propondo uma solução menos dependente de estruturas privadas, oferecendo um modelo mais democrático e acessível ao compartilhamento e consumo de conteúdos.
