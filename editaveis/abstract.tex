\begin{resumo}[Abstract]
 \begin{otherlanguage*}{english}
  The aim of this work is to develop a decentralized online learning application in order to guarantee greater privacy, security and autonomy in the sharing and consumption of digital educational content. The methodology used was a comparative analysis of traditional teaching platforms to understand their limitations and weaknesses, which are a consequence of the centralized architectural model. The results of the analysis show that decentralization can offer advantages over client-server architecture, providing resistance to censorship and greater control over data. As a result, a decentralized proposal was developed using technologies such as IPFS and OrbitDB, NextJS, NestJS and PostgreSQL.

   \vspace{\onelineskip}
 
   \noindent 
   \textbf{Key-words}: online learning. P2P. Decentralization. Centralization. IPFS. OrbitDB. Client-server architecture.
 \end{otherlanguage*}
\end{resumo}
