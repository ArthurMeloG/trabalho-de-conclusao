\chapter[Fundamentação Teórica]{Fundamentação Teórica}
\label{cap:fundamentacao}
\addcontentsline{toc}{chapter}{Fundamentação Teórica}

\section{Considerações iniciais}
Neste capítulo, apresenta-se o referencial teórico deste trabalho. Serão apresentadas as tecnologias utilizadas no desenvolvimento do sistema, sua arquitetura, bem como algumas ferramentas empregadas pela equipe durante o processo de construção do sistema.

\section{Descentralização e Tecnologias Distribuídas}
\subsection{Redes \textit{peer-to-peer}}
As redes \textit{peer-to-peer}, ou \textit{P2P}, são uma série de computadores de uma rede interligada em uma cadeia descentralizada, sendo que, cada um deles possui uma função equivalente, não havendo uma hierarquia entre eles, e portanto, sem a existência de um servidor central \cite{otton2007}. Essa é uma das características mais impactantes da descentralização: a ênfase na posse e controle dos dados e recursos pelos próprios usuários. Em um sistema verdadeiramente descentralizado, cada par (peer) tem um papel de participante igualitário \cite{oram2002}.

As razões pelas quais o uso de redes peer-to-peer é considerado atrativo podem ser resumidas em três principais pontos (BARCELLOS; GASPARY, 2006): 

\begin{itemize}
    \item Redes \textit{P2P} são escaláveis, ou seja, lidam eficientemente tanto com grupos pequenos quanto grupos grandes de participantes. 
    \item É possível depender mais do funcionamente dessas redes, já que não possuem um servidor central, e assim são mais resistentes a ataques ou censuras.
    \item Redes \textit{P2P} oferecem autonomia aos seus participantes, possibilitando que entrem e saiam da rede de acordo com sua vontade. 
\end{itemize}

\section{A tecnologia no suporte e complemento da educação}
Historicamente, o professor era a figura central das reuniões e salas de aula, sendo ele e os livros as principais fontes de informação. O aprendizado ocorria de forma predominantemente unilateral, em um modelo tradicional baseado na exposição de conteúdo pelo docente e na memorização por parte dos alunos \cite{unicep2024}.

A digitalização surgiu como uma força disruptiva, mudando por completo o modelo tradicional de educação. Ferramentas digitais, plataformas \textit{online}, e outros instrumentos passaram a ser participantes do processo de aprendizagem, facilitando o acesso ao conhecimento em uma larga escala. \cite{unicep2024}.

\section{Sistemas descentralizados no processo educacional}
Durante a pandemia causada pelo coronavírus (COVID-19), 2019, a impossibilidade de utilizar salas de aula causou inúmeras dificuldades, mas deixou um importante legado, a possibilidade de o processo de ensino poder acontecer em outros espaços, como o ambiente digital.

A digitalização quebra os limites de espaço e tempo, além de reduzir os custos relacionados à educação presencial, que envolvem uma grande estrutura para que possa acontecer, como mesas, cadeiras, prédios, computadores, funcionários e várias outras necessidades para que o ambiente suporte o processo educacional. Sendo assim, a digitalização da educação é um benefício disruptivo, trazendo inúmeras novas possibilidades.

Apesar dos avanços relacionados à digitalização de processos educacionais, esse movimento tem gerado, simultaneamente, a diversificação de ofertas e a concentração de poder. Isso ocorre porque as plataformas digitais adotam um modelo baseado no “domínio único”. Além disso, a plataformização da educação levanta diversas questões éticas: quem controla a privacidade e os dados de aprendizagem dos alunos? As grandes empresas de tecnologia serão responsáveis por definir os rumos da educação? Quem serão os autores desses novos ambientes digitais de aprendizagem? Em síntese, o poder tende a se concentrar nas mãos daqueles que detêm os dados educacionais, pois, com informações suficientes, é possível antecipar resultados, prever trajetórias ou até mesmo modificá-las \cite{valente2022}.

Com base no exposto, a descentralização de sistemas educacionais aparece como uma alternativa ao controle gerado pela arquitetura centralizada das empresas de tecnologia, garantindo autonomia para instrutores e alunos.

% \section{Produtos relacionados}
% A análise de produtos relacionados tem por finalidade apresentar soluções similares à deste trabalho. Essa análise é capaz de identificar padrões, limitações, oportunidades, além de compreender as soluções de forma prática e visual.  No contexto de aprendizagem online, existem diversas propostas, como a Udemy e a Hotmart, que já oferecem ambientes robustos e confiáveis, possuindo uma ampla base de usuários ativos. No entanto, estas plataformas dependem de uma infraestruturas centralizadas, que por consequência, estão sujeitas à todas as vantagens e desvantagens deste tipo de arquitetura.

\section{Diagrama de Gutenberg}

O desenvolvimento da aplicação web será baseado no diagrama de Gutenberg, que é o modelo que descreve princípios fundamentais de desing responsáveis por uma experiência de usuário intuitiva e eficiente. Dentre os diversos princípios utilizados, destacam-se para o desenvolvimento desse produto a {hierarquia visual}, \textit{consistência}, \textit{simplicidade} e \textit{leitura} que, quando combinados corretamente, constroem uma interface clara e funcional.

O princípio da \textit{hierarquia visual}, conforme proposto por \cite{lidwell2010}, refere-se à importância relativa dos elementos. Esses devem ser estruturados de forma intuitiva a fim de guiar o usuário pelas informações centrais. Isso pode ser alcançado através de tamanhos diferentes de fontes, cores  e contraste, tornando mais fácil a interação do usuário.

O princípio da \textit{consistência} afirma que elementos semelhantes devem ser tratados de maneira consistente em todo o projeto, ou seja, sem diferenças de desing que criem confusão no usuário. Isso aumenta a previsibilidade e acelera o aprendizado da interface, o que será essencial nesse projeto.

O princípio da \textit{simplicidade} auxilia na manutenção de uma interface limpa e objetiva partindo da organização dos elementos e funcionalidades e com isso, há a eliminação de elementos que sobrecarreguem o usuário com informações excessivas. Isso permite maior foco na interação e nas principais tarefas, aumentando a eficácia do produto.

Por fim, o princípio da \textit{leitura} destaca os padrões de leitura do ocidente. Esse padrão baseia-se no comportamento dos olhos humanos que tendem a começar a leitura do canto superior esquerdo de uma página, indo em direção ao limite direito, repetindo esse comportamento até o final da leitura \cite{lidwell2010}. Essa modelagem é fundamental na concepção de desings, pois dispõe informações a respeito da disposição dos componentes visuais da tela.

\section{Heurísticas de Nielsen}

As heurísticas de Nielsen são importantes princípios para a elaboração de interfaces de usuário. Elas são regras que tem por finalidade descobrir grandes e potenciais problemas da interface analisada \cite{nielsen1994usability} e, sendo assim, são poderosas ferramentas para o desenvolvimento e evolução de sítios web, ainda que os ambientes e interações tenham evoluído consideravelmente desde sua criação.

As normas estabelecidas por Nielsen estão intimamente ligadas aos métodos de avaliação de interface, que em sua a maioria, são baseados em engenharia cognitiva. Sua finalidade é o desenvolvimento de softwares que exijam do utilizador baixa carga cognitiva para sua aplicabilidade, sendo agradáveis e de fácil uso e aprendizado \cite{maciel2004avaliacao}.

\subsection{As heurísticas de Nielsen}

De acordo com a obra \cite{nielsen1994usability}, de Nielsen e Molich, são estas as heurísticas de usabilidade:

\begin{itemize}
    \item Visibilidade do estado: o sistema deve manter os usuários informados sobre seu atual estado através de /textit{feedback} apropriado, dentro de um tempo satisfatório.
    \item Correspondência entre sistema e realidade: o software deve ser baseado em palavras, símbolos e conceitos familiares ao usuário.
    \item Controle do usuário e liberdade: os usuários devem ser capazes de fazer e desfazer procedimentos, proporcionando uma saída fácil de qualquer estado indesejado.
    \item Consistência e padrões: siga convenções de usabilidade e padrões, a fim de não confundir os usuários do sistema.
    \item Prevenção de erros: o sistema deve prevenir a ocorrência de erros sempre que possível, por meio de restrições ou avisos.
    \item Reconhecimento em vez de memorização: os usuários devem ser capazes de reconhecer informações, e não memorizá-las.
    \item Flexibilidade e eficiência de uso: o software deve oferecer atalhos  a fim de possiblitar que usuários experientes operem em menor tempo.
    \item Estética e design minimalista: interfaces não devem possuir elementos desnecessárias ou raramente necessárias.
    \item Ajude os usuários a reconhecerem, diagnosticarem e recuperarem-se de erros: mensagens de erro devem ser claras, indicando o problema e sugerindo uma solução.
    \item Ajuda e documentação: deve ser fácil para os usuários acessarem a ajuda e a documentação (preferencimalmente curta).
\end{itemize}

\section{Processos de Design}
No processo de desenvolvimento de aplicações, é necessário considerar outros aspectos além do conhecimento técnico. Para que o produto entregue cumpra as expectativas e necessidades, é essencial adotar um pensamento sistêmico e uma abordagem estratégica. Nesse cenário, diversos métodos se propuseram a guiar a forma como o design é conduzido, garantindo que a experiência do usuário, a viabilidade técnica e os objetivos do negócio sejam equilibrados de maneira eficiente. Para esse trabalho de conclusão de curso, será utilizado o Desing Thinking.

\subsection{Design Thinking}
O Design Thinking é uma abordagem centrada no ser humano para a resolução de problemas, focada na inovação e na criação de soluções eficazes. Ao contrário dos processos tradicionais de desenvolvimento, que muitas vezes seguem um fluxo linear e estruturado, o Design Thinking propõe um modelo mais dinâmico e iterativo, no qual as soluções emergem a partir de ciclos de experimentação, teste e refinamento \cite{browndesignthinking}.

O processo de Desing Thinking pode ser descrito como um sistema de espaços interconectados, ao invés de uma estrutura rígida. Esse sistema pode ser dividido em 5 fases de acordo com Tim Brown:

\begin{itemize}
    \item Imersão: O objetivo da fase de imersão é identificar a necessidade do cliente, ou seja, o problema que deve ser resolvido. Nesse processo são comuns pesquisas, entrevistas e análise do comportamento do usuário a fim de entender suas necessidades e identificar oportunidades.
    \item Definição: A partir das informações coletadas na etapa anterior, é definido o problema principal a ser resolvido.
    \item Ideação: O objetivo da fase de ideação é a geração de ideias inovadoras, a partir das informações obtidas na fase anterior. 
    \item Prototipação: Construção de protótipos da solução para demonstrar as funcionalidades obtidas a partir das ideias coletadas.
    \item Testes: O objetivo da fase de testes é a validação das ideias. Sendo o Design Thinking um processo cíclico, é sempre possível revisitar as etapas anteriores para melhorar e refinar o produto final, garantindo que este esteja em conformidade com os anseios do usuário.
\end{itemize}


 