\chapter[Fundamentação Teórica]{Fundamentação Teórica}
\addcontentsline{toc}{chapter}{Fundamentação Teórica}

\section{O problema das aplicações centralizadas}

A arquitetura tradicional em que a internet se baseou, cliente-servidor, tem sido predominante em várias aplicações, como redes sociais, mensageiros, aplicações de busca. Apesar deste modelo ter se mostrado eficiente no decorrer do tempo, ele apresenta desafios significativos como a centralização dos dados. Segundo Wanner \cite{wanner2024}, problemas como a vulnerabilidade dos dados, a privacidade dos usuários e a censura são aspectos a serem avaliados nessa arquitetura que podem acarretar problemas de segurança e confiança. A forte tendência à centralização pode ser confirmada através do estudo da LogicMonitor \cite{logicmonitor2023} que confirma que com a pandemia COVID-19, a migração de dados das empresas para a nuvem foi acelerada 87\% e planeja seguir com esse processo. Diante dessa problemática, soluções como o \textit{InterPlanetary File System} (IPFS), bitcoin, nostr foram pensadas com o cerne na descentralização e democratização dos dados.

Neste projeto, o foco é desenvolver uma plataforma descentralizada para cursos e materiais educativos, utilizando o IPFS para garantir que os dados permaneçam sob o controle direto de seus criadores. A aplicação será responsável por gerenciar os dados dos usuários e no controle dos meios de pagamento através da rede  \textit{Lightning Network} \cite{lightningnetwork} para os conteúdos privados criptografados. Tanto os materiais públicos quanto os privados serão persistidos na rede IPFS. Dessa forma, mesmo que eventualmente essa aplicação saia de circulação ou seja bloqueada, os dados permanecerão disponíveis globalmente ao público.


\section{O \textit{InterPlanetary File System} (IPFS)}
O IPFS é um protocolo distribuído para armazenamento e compartilhamento de dados, que organiza arquivos em uma estrutura baseada em conteúdo, tornando a web mais eficiente e resistente a falhas \cite{ipfs2025}. Com ele é possível persistir os dados da aplicação entre os nós da rede. Isso o torna público, de fato, e replicável por quaisquer usuários que quiserem. Diferente do HTTP, que carrega arquivos de um servidor, o IPFS utiliza um modelo de endereçamento de conteúdo, em que os dados não desacoplados de sua localização e tratados a partir da hash única gerada pelo conteúdo do mesmo.

Entre suas principais características, destacam-se:

\begin{itemize}
    \item \textbf{Identificadores de Conteúdo (CIDs - Content Identifiers)}: cada conteúdo ou arquivo é associado a um endereço único baseado em um hash criptográfico, garantindo sua integridade e imutabilidade.
    
    \item \textbf{Rede Peer-to-Peer (P2P)}: milhares de nós interconectados facilitam a localização e recuperação dos dados distribuídos na rede.
    
    \item \textbf{Armazenamento em Cache}: os dados são armazenados temporariamente na memória cache dos nós, otimizando a largura de banda e melhorando a eficiência na redistribuição de conteúdos.
\end{itemize}

\subsection{OrbitDB: Banco de Dados Distribuído}

O OrbitDB é um banco de dados \textit{serverless} e ponto a ponto (\textit{peer-to-peer}), distribuído construído para atuar em aplicações descentralizadas. Ele usa o IPFS para o armazenamento dos dados e garante a sincronia dos dados entre os diferentes nós. Ele será utilizado no projeto para realizar a comunicação e persistência dos dados com a rede IPFS.

\subsection{Arquitetura de desenvolvimento web}


A arquitetura \textit{cliente-servidor} será adotada no desenvolvimento da aplicação, devido à sua robustez e flexibilidade no gerenciamento de interações entre usuários e servidores. Nesse modelo, os clientes fazem requisições a um servidor centralizado, que processa as solicitações e retorna as respostas adequadas. A utilização dessa abordagem oferece uma série de benefícios importantes para o desenvolvimento de sistemas distribuídos.


A arquitetura escolhida para o desenvolvimento da aplicação é a \textit{cliente-servidor}. Ela será adotada por conta de sua flexibilidade e robustez no gerenciamento de interações entre os usuários e os servidores. Essa parte da aplicação ficará encarregada pelo gerenciamento dos usuários, bem como pelo serviço de gerar as páginas estáticas que serão persistidas na rede ipfs com os dados gerados. O uso de uma abordagem híbrida concilia as vantagens da rede distribuída descritas acima com as vantagens dessa arquitetura:

\subsection{Principais vantagens do modelo Cliente-Servidor}

\begin{itemize}
    \item \textbf{Controle:} Apesar do cerne dessa aplicação ser a descentralização e democratização dos dados, a gestão dos usuários, bem como o controle do meio de pagamento via rede lightning foi pensado em uma aplicação centralizada. Com esse modelo é possível garantir a integridade e a consistência das informações. 
    Isso permite que a administração tenha maior controle, minimizando riscos de dados corrompidos ou acesso não autorizado\cite{coulouris2011}.
    
    \item \textbf{Escalabilidade:} A arquitetura cliente-servidor comporta um grande número de usuários. As aplicações nessa estrutura podem ser aprimoradas com mais recursos de acordo com a demanda \cite{coulouris2011}.
    
    \item \textbf{Desempenho e Eficiência:} A separação clara entre duas entidades corrobora com a performance da aplicação. O cliente foca na interação com o usuário e como esse dado será apresentado a ele. O servidor lida com processamento e armazenamento de dados. \cite{coulouris2011}.
    
    \item \textbf{Facilidade de Manutenção e Atualização:} Como o processamento centralizado ocorre no servidor, qualquer atualização ou manutenção do sistema pode ser realizada em um único ponto. Isso facilita a implementação de novos recursos, correção de falhas ou melhorias no sistema sem a necessidade de alterar os clientes \cite{coulouris2011}.
    
    \item \textbf{Flexibilidade e Compatibilidade:} Com essa arquitetura é possível distribuir a aplicação para uma gama de plataformas e alcançar um número maior de clintes, como aplicativos móveis, web ou desktop. \cite{coulouris2011}.
    
\end{itemize}

Dessa forma, a utilização desse modelo em conjunto com a abordagem descentralizada, contribuíra para o desenvolvimento de uma plataforma escalável, segura, eficiente e democrática.

\section{Desenvolvimento Cliente-Servidor}

A construção de uma aplicação percorre várias etapas, desde a concepção da ideia e validação até a prototipação e, seguidamente, codificação resultando em uma solução. Para atingir esse objetivo de maneira eficiente, é necessário escolher ferramentas e tecnologias que suportem  o tamanho e robustez que a aplicação se propõe. No projeto XXX, serão utilizadas tecnologias que atendam esses critérios.

Do lado do servidor, a aplicação será escrita a partir do framework NestJS, que se destaca por seu conjunto de ferramentas integradas capazes de comunicar-se com o banco de dados e construção de APIs rapidamente. É uma ferramenta contruída em JavaScript / TypeScript que proporcionará uma integração boa com o OrbitDB, haja visto que este é escrito na mesma linguagem.

Para o desenvolvimento do lado do cliente será adotado o nextJS, que atualmente é um framework amplamente utilizado por conta de suas facilidades no desenvolvimento web.

\subsection{NestJS}

NestJS é uma estrutura para construir aplicativos Node.js eficientes e escaláveis. Ele utiliza JavaScript progressivo, sendo construído com e oferecendo suporte total a TypeScript (mas ainda permitindo que os desenvolvedores codifiquem em JavaScript puro) e combina elementos de POO (Programação Orientada a Objetos), FP (Programação Funcional) e FRP (Programação Reativa Funcional) \cite{nestjs2025}. (Texto traduzido por autor).

\subsubsection*{Pontos Fortes do NestJS:}
\begin{itemize}
    \item \textbf{Modularidade}: Arquitetura que facilita a divisão da aplicação em módulos reutilizáveis.
    \item \textbf{Integração simples com bancos de dados}: Compatível com ORM populares, como TypeORM e Prisma.
    \item \textbf{Injeção de Dependência}: Simplifica o gerenciamento de dependências e aumenta a testabilidade.
    \item \textbf{Flexibilidade}: Suporta diversos paradigmas, como programação reativa e microsserviços.
\end{itemize}

\subsection{OrbitDB}

O \textbf{OrbitDB} é um banco de dados distribuído e descentralizado, construído sobre o protocolo \textit{IPFS} (\textit{InterPlanetary File System}). Sua arquitetura descentralizada permite alta disponibilidade e sincronização eficiente de dados entre nós. Como é escrito em \textit{JavaScript}, oferece uma integração perfeita com o NestJS.

\subsubsection*{Pontos Fortes do OrbitDB:}
\begin{itemize}
    \item \textbf{Descentralização}: Dispensa a necessidade de servidores centralizados.
    \item \textbf{Alta disponibilidade}: Suporta replicação automática e sincronização entre nós.
    \item \textbf{Desempenho escalável}: Ideal para aplicações distribuídas que requerem persistência de dados em tempo real.
    \item \textbf{Fácil integração}: Compatível com tecnologias baseadas em JavaScript.
\end{itemize}

\subsection{Next.js}

O \textbf{Next.js} é um framework \textit{React} para desenvolvimento de aplicações no lado do cliente. Ele oferece recursos como renderização no lado do servidor (\textit{Server-Side Rendering} - SSR) e geração de sites estáticos (\textit{Static Site Generation} - SSG), que permitem criar interfaces modernas com excelente desempenho.

\subsubsection*{Pontos Fortes do Next.js:}
\begin{itemize}
    \item \textbf{Renderização otimizada}: Suporte a SSR e SSG, melhorando o desempenho e SEO.
    \item \textbf{Rotas automáticas}: Simplifica a estrutura de navegação com rotas baseadas em arquivos.
    \item \textbf{Desenvolvimento eficiente}: Hot reloading e compilação automática durante o desenvolvimento.
    \item \textbf{API Routes}: Permite criar APIs no mesmo projeto do front-end.
    \item \textbf{Suporte ativo da comunidade}: Grande número de extensões e soluções prontas.
\end{itemize}

\subsection{PostgreSQL}

O \textbf{PostgreSQL} é um sistema de gerenciamento de banco de dados relacional de código aberto amplamente reconhecido por sua robustez, flexibilidade e conformidade com padrões SQL. Ele oferece suporte avançado a diversos tipos de dados e permite a criação de extensões personalizadas, tornando-o uma escolha ideal para aplicações modernas e escaláveis.

\subsubsection*{Pontos Fortes do PostgreSQL:}
\begin{itemize}
    \item \textbf{Suporte Avançado a Dados}: Compatível com tipos complexos, como JSON, arrays, e dados geoespaciais (\textit{PostGIS}).
    \item \textbf{Alta Conformidade com Padrões SQL}: Facilita a portabilidade de aplicações entre diferentes bancos de dados.
    \item \textbf{Extensibilidade}: Suporte a funções definidas pelo usuário, tipos de dados customizados e linguagens de programação adicionais.
    \item \textbf{Desempenho e Escalabilidade}: Otimizações para consultas complexas e grandes volumes de dados, com suporte a replicação e particionamento.
    \item \textbf{Segurança}: Autenticação robusta, criptografia e controle granular de permissões.
    \item \textbf{Comunidade Ativa}: Grande quantidade de documentação e suporte da comunidade global.
\end{itemize}
