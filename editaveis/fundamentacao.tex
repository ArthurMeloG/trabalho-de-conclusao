\chapter[Fundamentação Teórica]{Fundamentação Teórica}
\addcontentsline{toc}{chapter}{Fundamentação Teórica}

A fundamentação teórica tem como objetivo apresentar os principais conceitos e tecnologias que embasam o desenvolvimento deste trabalho. São abordados temas como assistentes virtuais, sistemas de mensageria, automação de atendimento, bases de conhecimento e ferramentas utilizadas no projeto, bem como aspectos relacionados à organização institucional na universidade de brasília.

\section{Organização Institucional da UnB}

A Universidade de Brasília (UnB) é uma instituição federal de ensino superior localizada na capital do país, fundada em 1962, apenas dois anos após a inauguração de Brasília. Atualmente, a universidade conta com quatro campi: o principal, situado no Plano Piloto (Campus Darcy Ribeiro), e três localizados em regiões administrativas do Distrito Federal — Ceilândia, Gama e Planaltina. Juntas, essas unidades oferecem um total de 76 cursos de graduação, atendendo a uma comunidade acadêmica ampla e diversa. 

A UnB possui uma estrutura organizacional formalizada, conforme o organograma \cite{organogramaUnB2017}. No topo dessa estrutura está o Conselho Universitário (CONSUNI), órgão máximo deliberativo responsável por decisões estratégicas, tais como criação e extinção de cursos, aprovação do Regimento Geral e dos regimentos internos de Unidades Acadêmicas, Órgãos Complementares e Centros.

Abaixo do CONSUNI encontra-se a Reitoria, conduzida pelo Reitor, com o apoio da Vice‑Reitoria, responsáveis pela administração executiva da instituição, coordenando também os diversos Decanatos, que cuidam das áreas de Planejamento, Extensão, Pesquisa, Administração e outras funções institucionais.

Posteriormente, encontram-se as Unidades Acadêmicas, compostas por Institutos e Faculdades, cada um subdividido em Departamentos que organizam o ensino, a pesquisa, e as atividades de extensão em áreas específicas do conhecimento. Essas unidades têm autonomia para elaborar normativas internas, complementares às diretrizes gerais definidas em nível central.

Dessa forma, parte das normas institucionais são centralizadas — como políticas de matrícula, calendário acadêmico e procedimentos gerais — enquanto outras variam conforme a Unidade Acadêmica, como regras para estágios, validações curriculares e trâmites de colação de grau. Essas informações, embora públicas, costumam estar dispersas em editais, resoluções e manuais administrativos, geralmente em formato PDF, e organizadas por cada instância responsável.


\section{Assistentes Virtuais e Chatbots}

Os assistentes virtuais, também conhecidos como chatbots, são sistemas computacionais capazes de interagir com seres humanos por meio de linguagem natural, geralmente em plataformas de mensagens instantâneas. Seu principal objetivo é automatizar tarefas repetitivas, responder dúvidas e fornecer informações de forma rápida e acessível.

Segundo \cite{russell2010artificial}, chatbots são uma das aplicações mais populares da inteligência artificial aplicada à linguagem natural, utilizando regras ou modelos de aprendizado para compreender e responder ao usuário. No contexto acadêmico, sua adoção tem crescido, oferecendo suporte a alunos em processos como matrícula, calendário acadêmico, prazos e requisitos administrativos.

\section{Impacto dos assistentes virtuais na industria de software}



\section{Aplicações de Bots em Ambientes Educacionais}

Diversas instituições de ensino têm adotado bots como forma de ampliar o atendimento e oferecer suporte ao corpo discente. Soluções como o "BIA" da Universidade Federal da Bahia e o "DudaBot" da Universidade de São Paulo são exemplos de iniciativas que visam tornar mais amigável o acesso a informações institucionais por meio do WhatsApp e do Telegram.

Esses bots geralmente se conectam a bases de dados ou documentos previamente estruturados, permitindo a extração automatizada de respostas e a personalização da experiência do usuário. Além disso, a integração com APIs de mensageiros facilita a entrega dessas informações diretamente nos dispositivos móveis dos estudantes.

\section{Sistemas de Mensageria: WhatsApp e Telegram}

O WhatsApp e o Telegram são as duas plataformas de mensageria mais populares entre os jovens universitários brasileiros. Ambas oferecem APIs que permitem a criação de bots automatizados, com diferentes níveis de acesso e personalização.

\subsection{WhatsApp Business API}

A API oficial do WhatsApp Business permite a criação de bots corporativos, mas seu uso exige homologação junto ao Facebook (Meta) e possui restrições quanto ao envio de mensagens ativas. É ideal para atendimentos estruturados e fluxos com objetivos claros.

\subsection{Telegram Bot API}

O Telegram, por outro lado, oferece uma API mais aberta e de fácil integração. Permite a criação de bots gratuitos com funcionalidades variadas, como menus interativos, envio de arquivos e tratamento de comandos específicos. Isso o torna uma excelente opção para prototipagem e testes acadêmicos.

\section{Tecnologias e Ferramentas Utilizadas}

O desenvolvimento do assistente virtual proposto será realizado com base em tecnologias modernas e acessíveis. Dentre as ferramentas e linguagens consideradas, destacam-se:

\begin{itemize}
    \item \textbf{Python}: Linguagem de programação utilizada para o desenvolvimento do bot e do backend.
    \item \textbf{Flask / FastAPI}: Frameworks web leves e ideais para APIs e microserviços.
    \item \textbf{MongoDB ou SQLite}: Banco de dados para armazenamento da base de conhecimento.
    \item \textbf{Telegram Bot API}: Plataforma principal para teste e validação do protótipo.
\end{itemize}

O uso dessas tecnologias visa garantir baixo custo de desenvolvimento, facilidade de manutenção e rápida implantação em ambiente universitário.
