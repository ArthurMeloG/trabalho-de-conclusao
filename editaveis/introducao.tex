\chapter[Introdução]{Introdução}

\addcontentsline{toc}{chapter}{Introdução}

\section{Contexto Tecnológico}

A Web (World Wide Web) foi concebida como uma plataforma aberta e acessível para todos, promovendo liberdade de acesso e compartilhamento de informações de qualquer lugar e hora. Em novembro de 2007, 5 anos após o surgimento da Web 2.0, a WWW inicia suas operações com uma série de objetivos, como a criação de um fórum para a comunidade interessada em padrões web, tradução para português de conteúdos que fossem do interesse regional, entre outros \cite{vieira2014}.

Desde então, a W3C tem evoluído cada vez mais, transformando não só como as informações são acessadas e compartilhadas, mas também o estilo de vida dos seres humanos, uma vez que afetou os padrões de interação social, comunicação, consumo, trabalho e muitos outros fatores, em todo o mundo.

A Web possui vantagens, desvantagens e características particulares. Uma das principais peculiaridades é que seu funcionamento está atrelado ao modelo cliente-servidor, e como consequência, a maioria das máquinas que participam do processo de compartilhamento e acesso de informações são apenas coadjuvantes, consumindo serviços de outras máquinas servidoras \cite{rocha2004p2p}. Dentre as desvantagens, observa-se o risco de centralização, onde os usuários estão sob uma entidade central, que pode gerar preocupações, como privacidade e censura. 

Para contrapor tais limitações e desvantagens, o presente trabalho tem por finalidade desenvolver uma aplicação segura e descentralizada através do paradigma P2P (peer-to-peer), para aprendizado comunitário de assuntos gerais, onde o proprietário do material possui total controle sobre seu conteúdo, sendo capaz de acessá-lo e distribuí-lo da forma como quiser, protegendo o material contra acessos não autorizados e manipulações, promovendo segurança e confiança no ambiente online..

\section{Problema}

Grande parte dos conteúdos educativos da Web estão disponíveis em plataformas centralizadas, como, por exemplo, blogs e comunidades, onde seus colaboradores desenvolvem o conteúdo em diversos formatos, e então o publicam. No entanto, como consequência da utilização do modelo tradicional cliente-servidor, as plataformas podem gerar perda de autonomia sobre o material publicado e dependência da infraestrutura da empresa responsável. Além disso, problemas como vulnerabilidade dos dados, a privacidade dos usuários e a censura também são algumas preocupações nesse tipo de arquitetura, podendo ocasionar problemas de confiabilidade e segurança \cite{wanner2024}. Ainda por cima, é simples concluir que, por várias destas questões, a resultante alcançada é a indisponibilidade do conteúdo, comprometendo o acesso de usuários que dependem dessas informações. 

A centralização também levanta preocupações com a privacidade dos dados, uma vez que as plataformas podem rastrear e utilizar informações dos usuários para fins comerciais sem consentimento explícito \cite{beiro2020}. Assim, surge a necessidade de explorar alternativas inovadoras e descentralizadas que garantam maior resiliência, privacidade e controle aos criadores e consumidores de conteúdos educativos, redefinindo a forma como o aprendizado online acontece.

\section{Objetivos}

A descentralização surge como uma alternativa inovadora, promovendo resiliência, maior privacidade e autonomia dos usuários ao permitir que dados sejam distribuídos e armazenados em redes ponto a ponto, sem depender de servidores centralizados. Com isso, o projeto Learn Chain foi arquitetado para garantir a integridade e privacidade dos dados sensíveis do material didático. 

\subsection{Objetivo Geral}

Desenvolver uma plataforma de aprendizado comunitário que utilize uma arquitetura moderna e segura, garantindo integridade e privacidade dos dados sensíveis através de técnicas de criptografia e tecnologias descentralizadas.

\subsection{Objetivos Específicos}

\begin{itemize}
    \item Implementar uma arquitetura back-end eficiente e escalável utilizando o framework NestJS integrado com banco de dados Postgres e outras soluções como OrbitDB e Helia, para o gerenciamento seguro dos dados.
    \item Garantir a privacidade e segurança dos dados dos usuários por meio de técnicas de criptografia, armazenando apenas informações necessárias e sensíveis de forma protegida.
    \item Utilizar o IPFS (InterPlanetary File System) para o armazenamento descentralizado de materiais de curso, otimizando a disponibilidade e a proteção dos arquivos.
    \item Implementar um sistema de controle de acesso robusto que permita o compartilhamento seletivo de materiais com base nos níveis de permissão dos usuários, assegurando que conteúdos privados sejam acessíveis apenas para os usuários autorizados.
    \item Fornecer uma interface amigável e responsiva para os administradores e alunos, desenvolvida com NextJS, que facilite o acesso aos cursos e materiais disponibilizados na plataforma.
    \item Documentar e testar a solução para validar a conformidade da plataforma com os requisitos funcionais e de segurança estabelecidos.
\end{itemize}
  