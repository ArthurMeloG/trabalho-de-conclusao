\chapter[Introdução]{Introdução}
\label{cap:introducao}
\addcontentsline{toc}{chapter}{Introdução}

O objetivo principal da indtrodução é dar um contexto geral sobre o tema abordado neste Trabalho de Conclusão de Curso (TCC). Para isso, será apresentado um panorama geral da metodologia científica utilizada, para obtenção dos resultados esperados. Além disso, será evidenciado os principais problemas que motivaram a realização deste estudo, destacando sua relevância e impacto dentro do contexto em que estão inseridos. 

\section{Contexto Tecnológico}

A Web (\textit{World Wide Web}) foi concebida como uma plataforma aberta e acessível para todos, promovendo liberdade de acesso e compartilhamento de informações de qualquer lugar e hora. Em novembro de 2007, 5 anos após o surgimento da Web 2.0, a WWW inicia suas operações com uma série de objetivos, como a criação de um fórum para a comunidade interessada em padrões web, tradução para português de conteúdos que fossem do interesse regional, entre outros \cite{vieira2014}.

Desde então, a W3C tem evoluído cada vez mais, transformando não só como as informações são acessadas e compartilhadas, mas também o estilo de vida dos seres humanos, uma vez que afetou os padrões de interação social, comunicação, consumo, trabalho e de muitos outros fatores em todo o mundo.

A Web possui vantagens, desvantagens e características particulares. Uma de suas principais peculiaridades, é que seu funcionamento está atrelado ao modelo cliente-servidor e, como consequência, a maioria das máquinas que participam do processo de compartilhamento e acesso de informações são apenas coadjuvantes, consumindo serviços de outras máquinas servidoras \cite{rocha2004p2p}. Já no que concerne às desvantagens, observa-se o risco de centralização, em que os usuários estão sob uma entidade central, o que pode gerar preocupações referentes à privacidade e censura. 

Para contrapor tais limitações e desvantagens, o presente trabalho tem por finalidade desenvolver uma aplicação segura e descentralizada, para aprendizado comunitário de assuntos gerais, onde o proprietário do material possui total controle sobre seu conteúdo, sendo capaz de acessá-lo e distribuí-lo da forma como quiser, protegendo o conteúdo contra acessos não autorizados e manipulações, promovendo segurança e confiança no ambiente online. Isso será feito por meio da utilização do modelo arquitetural cliente-servidor, atrelado à sua arquitetura, tecnologias cujo funcionamento baseia-se no paradigma \textit{Peer-to-peer}.

\section{Questão de Pesquisa}
Diante desse contexto, o ponto chave que orienta este trabalho é:

Como criar uma solução de software, para aprendizado online, sem a dependência de plataformas e serviços centralizados?

\section{Problema e Justificativa}

Grande parte dos conteúdos educativos da Web estão disponíveis em plataformas centralizadas, como, blogs e comunidades, onde seus colaboradores desenvolvem o conteúdo em diversos formatos, e então o publicam. No entanto, como consequência da utilização do modelo tradicional cliente-servidor, as plataformas podem gerar perda de autonomia sobre o material publicado e dependência da infraestrutura da empresa responsável. Além disso, problemas como vulnerabilidade dos dados, a privacidade dos usuários e a censura também são algumas preocupações nesse tipo de arquitetura, podendo ocasionar problemas de confiabilidade e segurança \cite{wanner2024}. Nesse sentido, cabe ressaltar que, por essas e outras questões, a resultante alcançada é a indisponibilidade do conteúdo, o que compromete o acesso de usuários que dependem dessas informações. 

A centralização também levanta preocupações com a privacidade dos dados, uma vez que as plataformas podem rastrear e utilizar informações dos usuários para fins comerciais sem consentimento explícito \cite{beiro2020}. Assim, surge a necessidade de explorar alternativas inovadoras e descentralizadas que garantam maior resiliência, privacidade e controle aos criadores e consumidores de conteúdos educativos, redefinindo a forma como o aprendizado online acontece ordienamente.

A partir dessas questões, identifica-se a oportunidade de atuar nesse cenário, construindo uma aplicação, com o uso da tecnologia descentralizada como ferramenta principal, que seja capaz de oferecer um sistema de aprendizado online.

\section{Objetivos}

A descentralização surge como uma alternativa inovadora, promovendo resiliência, maior privacidade e autonomia dos usuários ao permitir que dados sejam distribuídos e armazenados em redes ponto a ponto, sem depender de servidores centralizados. Com isso, o projeto Learn Chain foi arquitetado para garantir a integridade e privacidade dos dados sensíveis do material didático. 

\subsection{Objetivo Geral}

Desenvolver uma plataforma de aprendizado comunitário que utilize uma arquitetura moderna e segura, garantindo integridade e privacidade dos dados sensíveis através de técnicas de criptografia e tecnologias descentralizadas.

\subsection{Objetivos Específicos}

\begin{itemize}
    \item Implementar uma arquitetura back-end eficiente e escalável utilizando o framework NestJS integrado com banco de dados Postgres e outras soluções como OrbitDB e Helia, para o gerenciamento seguro dos dados.
    \item Garantir a privacidade e segurança dos dados dos usuários por meio de técnicas de criptografia, armazenando apenas informações necessárias e sensíveis de forma protegida.
    \item Utilizar o IPFS (\textit{InterPlanetary File System}) para o armazenamento descentralizado de materiais de curso, otimizando a disponibilidade e a proteção dos arquivos.
    \item Implementar um sistema de controle de acesso robusto que permita o compartilhamento seletivo de materiais com base nos níveis de permissão dos usuários, assegurando que conteúdos privados sejam acessíveis apenas para os usuários autorizados.
    \item Fornecer uma interface amigável e responsiva para os administradores e alunos, desenvolvida com NextJS, que facilite o acesso aos cursos e materiais disponibilizados na plataforma.
    \item Documentar e testar a solução para validar a conformidade da plataforma com os requisitos funcionais e de segurança estabelecidos.
\end{itemize}

\section{Metodologia}

Durante a elaboração deste trabalho serão empregadas duas metodologias, sendo uma de pesquisa, e outra de desenvolvimento.

\subsection{Metodologia de Pesquisa}

A metodologia de pesquisa descreve os métodos utilizados para construir a fundamentação teórica deste projeto. Com ela, é possível classificar este trabalho quanto à sua natureza, finalidade, forma de abordagem, objetivos e procedimentos técnicos.

Com isso, as devidas classificações podem ser observadas na Tabela 1, que será melhor explorada posteriormente.

\setlength{\extrarowheight}{5pt}

\begin{table}
    \centering
    \caption{Classificações de Metodologia de pesquisa}
    \begin{tabular}{|l|l|}
        \hline
        \textbf{Classificação}            & \textbf{Tipo de Pesquisa}\\ 
        \hline
        Quanto à finalidade               & Pesquisa tecnológica \\ 
        \hline
        Quanto à natureza                 & Pesquisa experimental \\ 
        \hline
        Quanto à forma de abordagem       & Pesquisa qualitativa \\
        \hline
        Quanto aos objetivos              & Pesquisa exploratória \\
        \hline
        Quanto aos procedimentos técnicos & Pesquisa bibliográfica \\        
        \hline
    \end{tabular}
    \label{tab:tipo_pesquisa}
    \vspace{5mm} \\ 
    {\footnotesize Fonte: Autores} 
\end{table}

\subsection{Metodologia de desenvolvimento}

Para a elaboração do projeto serão utilizadas metodologias ágeis, apoiando-se em técnicas de desenvolvimento como \textit{Extreme Programming} (XP), kanban, além da utilização de ferramentas de DevOps para uso de containers como Docker. A partir disso é possível estabelecer um ambiente de trabalho previsível e facilitado para possíveis mudanças e refatorações do projeto. Ademais, serão empregadas algumas técnicas do Scrum em conjunto com o XP servirão de base para o Backlog.

\subsection{Estrutura do trabalho}

Este trabalho está disposto da seguinte maneira:

Capítulo \ref{cap:introducao}: Introdução: Este capítulo faz uma levantamento geral das ideias, ferramentas, metodologias utilzidas para o desenvolvimento do projeto.

Caítulo \ref{cap:fundamentacao}: Fundamentação Teórica: Aqui será aprofundado o etendimento das tecnologias utilizadas bem como a razão de uso de cada uma delas no contexto do desenvolvimento distribuído.

Capítulo \ref{cap:metodologia}: Proposta de Solução: Este capítulo ajuda a entender as metodologias de pesquisa e desenvolvimento para o trabalho, e detalha os requisitos, a arquitetura e as atividades que serão desenvolvidas no fluxo de trabalho do desenvolvimento do software.

Capítulo \ref{cap:consideracoes}: Considerações Finais: Este capítulo finaliza a o Trabalho de Conclusão de Curso (TCC 1).