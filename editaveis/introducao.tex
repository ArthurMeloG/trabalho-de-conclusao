\chapter[Introdução]{Introdução}
\addcontentsline{toc}{chapter}{Introdução}

O acesso à informação é um fator essencial para a boa experiência acadêmica e para a tomada de decisões por parte dos estudantes universitários. Na Universidade de Brasília (UnB), assim como em diversas outras instituições de ensino superior, informações institucionais relevantes — como prazos acadêmicos, procedimentos administrativos, normas para trancamento de disciplinas, emissão de documentos, entre outros — encontram-se frequentemente dispersas em documentos oficiais, editais e regulamentos que, por vezes, são de difícil acesso ou compreensão para os alunos.

Essa dificuldade no acesso a informações básicas pode gerar insegurança, atrasos em processos acadêmicos e até prejuízos no percurso estudantil. Considerando esse cenário, percebe-se uma oportunidade de melhoria na forma como essas informações são disponibilizadas e consultadas pelos discentes. Com o avanço das tecnologias de comunicação, especialmente o uso massivo de aplicativos de mensagens como WhatsApp e Telegram, torna-se viável e relevante a criação de soluções tecnológicas que se integrem ao cotidiano dos estudantes para facilitar esse acesso.

Este trabalho propõe o desenvolvimento de uma ferramenta digital, na forma de um bot integrado ao WhatsApp e ao Telegram, que terá como objetivo auxiliar os estudantes da UnB no acesso rápido e direto a informações institucionais relevantes. O bot será alimentado por uma base de dados construída a partir da extração e estruturação de conteúdos presentes em documentos oficiais da universidade.

A proposta visa, portanto, contribuir para a melhoria da comunicação entre a instituição e seus estudantes, promovendo autonomia, agilidade e clareza no acesso às informações. Para fundamentar a relevância da proposta, uma pesquisa foi realizada com estudantes da universidade, cujos resultados subsidiarão a etapa de desenvolvimento da ferramenta.

Este trabalho está estruturado da seguinte forma: na seção de Referencial Teórico, são apresentados os conceitos e tecnologias envolvidos na proposta; a seção de Metodologia descreve os passos adotados para a validação, modelagem e construção da solução; por fim, a Conclusão apresenta os resultados esperados, considerações finais e possibilidades de continuidade do projeto.

\section{Objetivos}

Esta seção tem como finalidade apresentar de forma clara e organizada os propósitos centrais deste trabalho de conclusão de curso. Os objetivos definidos servem como diretrizes para a condução da pesquisa e do desenvolvimento do sistema proposto, permitindo delimitar o escopo da solução e orientar as etapas metodológicas. Para isso, os objetivos estão divididos em objetivo geral, que expressa a meta principal do projeto, e objetivos específicos, que detalham as etapas e metas intermediárias necessárias para alcançar o resultado pretendido.

\subsection{Objetivo Geral}

Desenvolver uma ferramenta digital, na forma de um bot para WhatsApp e Telegram, com o objetivo de auxiliar estudantes da Universidade de Brasília (UnB) no acesso rápido e facilitado a informações institucionais extraídas de documentos oficiais, como editais, regulamentos e normativas acadêmicas.

\subsection{Objetivos Específicos}

\begin{itemize}
    \item Identificar as principais dificuldades enfrentadas pelos estudantes da UnB no acesso a informações institucionais, por meio da aplicação de uma pesquisa exploratória;
    
    \item Levantar, organizar e estruturar conteúdos relevantes a partir de documentos oficiais da universidade, construindo uma base de dados de apoio ao bot;
    
    \item Projetar a arquitetura da ferramenta, definindo fluxos de conversa e mecanismos de busca e resposta;
    
    \item Implementar um bot funcional nas plataformas WhatsApp e Telegram, com integração à base de dados construída;
    
    \item Validar a eficácia da ferramenta por meio de testes com usuários reais e coletar feedbacks para possíveis melhorias.
\end{itemize}

