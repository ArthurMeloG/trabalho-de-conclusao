\chapter[Introdução]{Introdução}
\addcontentsline{toc}{chapter}{Introdução}

O acesso à informação é um fator essencial para a boa experiência acadêmica dos estudantes universitários. Na Universidade de Brasília (UnB), informações institucionais relevantes, como prazos acadêmicos, procedimentos administrativos, normas para trancamento de disciplinas, emissão de documentos, entre outros, encontram-se frequentemente dispersas em documentos oficiais, editais e regulamentos que, por vezes, são numerosos e complexos.                                                                        

A incerteza e a demora nos estudos na vida acadêmica podem surgir da complexidade em obter informações básicas, resultando em prejuízos no percurso estudantil. Assim, nota-se a chance de aprimorar a forma como os alunos encontram e acessam os dados. Com o uso generalizado de aplicativos de comunicação Mobile, é válido a ideia de criar soluções tecnológicas que se encaixem em software já conhecidos, como WhatsApp e Telegram, adaptando-se na rotina dos discentes, e assim simplificando o acesso.

% REVIEW 2 - colocar uma possível referência aqui para ajudar a embasar que estudantes por vezes tem dificuldade em acessar informações.

Este trabalho tem por finalidade o desenvolvimento de um sistema na forma de um bot integrado ao WhatsApp e ao Telegram, que terá como objetivo auxiliar os estudantes da UnB no acesso simplificado a informações institucionais administrativas. Para a construção do Bot, será utilizada uma base de dados construída a partir da extração e estruturação de documentos oficiais da UnB.

A proposta visa, portanto, melhorar a comunicação entre a instituição e seus estudantes, promovendo autonomia, agilidade e clareza no acesso às informações. Para fundamentar a relevância da proposta, uma pesquisa foi realizada com estudantes da universidade, cujos resultados contribuirão no desenvolvimento do software, principalmente na compreensão dos requisitos e espectativas da aplicação.

Este trabalho está estruturado da seguinte forma: seção de introdução, a qual apresenta-se a solução proposta e contextualiza o leitor acerca do tema; a seção de Referencial Teórico, em que são apresentados os conceitos e tecnologias envolvidos na proposta; a seção de Metodologia, que descreve os passos adotados para a validação, modelagem e construção da solução; por fim, a Conclusão apresenta os resultados esperados, considerações finais e possibilidades de continuidade do projeto.

\section{Objetivos}

Esta seção tem como finalidade apresentar de forma clara e organizada os propósitos centrais deste trabalho de conclusão de curso. Os objetivos definidos servem como diretrizes para a condução da pesquisa e do desenvolvimento do sistema proposto, permitindo delimitar o escopo da solução e orientar as etapas metodológicas. Para isso, os objetivos estão divididos em objetivo geral, que expressa a meta principal do projeto, e objetivos específicos, que detalham as etapas e metas intermediárias necessárias para alcançar o resultado pretendido.

\subsection{Objetivo Geral}

Desenvolver uma ferramenta digital, na forma de um bot para WhatsApp e Telegram, com o objetivo de auxiliar estudantes da Universidade de Brasília (UnB) no acesso rápido e facilitado a informações institucionais extraídas de documentos oficiais.

\subsection{Objetivos Específicos}

\begin{itemize}
    \item Identificar as principais dificuldades enfrentadas pelos estudantes da UnB no acesso a informações institucionais, por meio da aplicação de uma pesquisa exploratória;
    
    \item Levantar, organizar e estruturar conteúdos relevantes a partir de documentos oficiais da universidade, construindo uma base de dados de apoio ao bot;
    
    \item Projetar a arquitetura da ferramenta, definindo fluxos de conversa e mecanismos de busca e resposta;
    
    \item Implementar um bot funcional nas plataformas WhatsApp e Telegram, com integração à base de dados construída;
    
    \item Validar a eficácia da ferramenta por meio de testes com usuários reais e coletar feedbacks para possíveis melhorias.
\end{itemize}

