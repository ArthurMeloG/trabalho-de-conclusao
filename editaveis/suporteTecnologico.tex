\chapter[Elementos do Texto]{Suporte Tecnológico}

\section{Nostr e IPFS: Fundamentação da escolha}
A escolha das ferramentas IPFS e Nostr foi baseada na necessidade de alcançar um sistema descentralizado e seguro, capaz não só de armazenar os dados, mas também realizar a troca de mensagens entre participantes de forma descentralizada. Além disso, essas tecnologias são amplamente utilizadas pela comunidade para alcançar os objetivos que a equipe almeja com o desenvolvimento do presente trabalho, o que reforça a decisão de utilizar tais tecnologias no projeto.

\section{NextJS e NestJS}
% As tecnologias NextJS e NestJS emergiram como componentes fundamentais no desenvolvimento deste projeto, alinhando-se aos requisitos levantados pelos participantes. Como observado na página introdutória ao Next.js é um framework React projetado para construir aplicativos web de alta qualidade. Ele simplifica o processo de desenvolvimento ao fornecer otimizações integradas, renderização do lado do servidor e roteamento fácil. Com recursos como busca de dados, otimização automática de imagens e opções de renderização flexíveis, ajudando os desenvolvedores a criar sites rápidos, escaláveis ​​e eficientes (citacao do next). NestJS é uma estrutura progressiva do Node.js projetada para construir aplicativos escaláveis, confiáveis ​​e eficientes do lado do servidor. Ele oferece modularidade, escalabilidade e ferramentas poderosas, como injeção de dependência, suporte a TypeScript e um ecossistema rico. É ideal para desenvolver APIs REST, APIs GraphQL, microsserviços e aplicativos em tempo real, com foco em manter uma arquitetura limpa e reduzir a complexidade. Confiável por empresas, o NestJS também foi projetado para ser extensível e flexível, tornando-o adequado para vários tipos de aplicativos (citação nest js).

\section{Git e GitHub Pages: Controle e Documentação}
