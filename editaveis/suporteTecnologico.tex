\chapter[Elementos do Texto]{Suporte Tecnológico}

\section{Nostr e IPFS: Fundamentação da escolha}
A escolha das ferramentas IPFS e Nostr foi baseada na necessidade de alcançar um sistema descentralizado e seguro, capaz não só de armazenar os dados, mas também realizar a troca de mensagens entre participantes de forma descentralizada. Além disso, essas tecnologias são amplamente utilizadas pela comunidade para alcançar os objetivos que a equipe almeja com o desenvolvimento do presente trabalho, o que reforça a decisão de utilizar tais tecnologias no projeto.

\section{NextJS e NestJS}
As tecnologias NextJS e NestJS emergiram como componentes fundamentais no desenvolvimento deste projeto, alinhando-se aos requisitos levantados pelos participantes. Como descrito na página de apresentação da ferramente, o Next.js é um framework React projetado para construir aplicativos web de alta qualidade , simplificando o processo de desenvolvimento ao fornecer otimizações integradas, renderização do lado do servidor e roteamento fácil \cite{nextjs2025}. 

Analisando simultaneamente, o NestJS é uma estrutura progressiva baseada em Node.js, ideal para desenvolver aplicativos do lado do servidor escaláveis, confiáveis e eficientes \cite{nestjs2025}, além de oferecer suporte nativo a linguagem TypeScript. Essas características tornam o NestJS particularmente adequado para a construção de APIs REST, que será o padrão utilizado no presente projeto.

\section{Armazenamento e persistência de dados}

\subsection{PostgreSQL}
Conhecido por ser um dos sistemas de armazenamento de dados mais seguros e confiáveis da atualidade, o PostgreSQL está sendo utilizado para armazenar os usuários da plataforma, além dos metadados dos materiais de aprendizagem da plataforma. Além das características de confiabilidade, sua escolha também foi influenciada pela experiência dos membros desenvolvedores com tal tecnologia.

\subsection{IPFS e OrbitDB}
O IPFS, jutantamente com o OrbitDB desempenham papéis fundamentais no desenho do projeto, permitindo a descentralização no armazenamento e tratamento de dados. O IPFS, além do fato de ser bastante utilizado pela comunidade de desenvolvimento para a construção de aplicações descentralizadas, oferece uma grande capacidade de distribuir e armazenar arquivos de maneira globalmente acessível, e por isso foi escolhido como complemento arquitetural do trabalho. De forma conjunta, o OrbitDB complementa o sistema com uma solução de armazenamento distribuído baseada no protocolo IPFS.


\section{Figma}
O Figma é uma ferramenta de prototipação colaborativa, que possui um fluxo de trabalho criado para desenvolvedores de Software, e tem como finalidade facilitar as tarefas de design com uma interface simples e bastante intuitiva \cite{figma2025}.

A utilização do Figma foi fundamental para o sucesso do projeto, onde a equipe teve a capacidade de desenvolver o protótipo de alta fidelidade, construindo designs interativos e detalhados. O time, trabalhando simultaneamente, fez uso da tecnologia para a construção e simulação de fluxos de navegação, testando a experiência do usuário de forma prática eficiente. Além disso, o recurso de entrega de ativos foi um facilitador para a integração entre design e desenvolvimento, tornando-se assim, uma ferramenta essencial para a elaboração visual do trabalho.
