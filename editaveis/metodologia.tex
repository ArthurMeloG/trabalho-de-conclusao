\chapter[Metodologia]{Metodologia}

Este capítulo tem como finalidade detalhar a abordagem metodológica e de desenvolvimento adotada pela equipe para construir o presente trabalho, delineando os passos previstos para a conclusão do sistema.

\section{Metodologia de Pesquisa}

Nesta seção serão apresentadas as características dos tipos de pesquisa, conforme introduzido no Capítulo 1. A análise deve ser capaz de classificar este trabalho quanto à sua natureza, final idade, forma de abordagem, objetivos e procedimentos técnicos.

\subsection{Tipo de Pesquisa (quanto à finalidade)}
O objetivo deste trabalho é a produção de conhecimento científico através de uma aplicação prática, a saber, um software educacional distribuído. Por isso, segundo Fontelles et al. (2009), o presente trabalho classifica-se como um tipo de pesquisa aplicada ou tecnológica \cite{fontelles2009}.

\subsection{Tipo de Pesquisa (quanto à natureza)}
A presente pesquisa realizada é de natureza experimental, visto que o foco está no teste da eficácia do aplicativo de aprendizado descentralizado. O objetivo é avaliar como a arquitetura deste sistema pode impactar a disponibilidade, segurança e privacidade dos materiais e usuários da plataforma \cite{fontelles2009}.

\subsection{Tipo de Pesquisa (quanto à forma de abordagem)}
A presente pesquisa será de abordagem qualitativa, tendo em vista que busca compreender as necessidades, experiências e desejos de usuários que dependem da \textit{Internet} como meio de estudos \cite{fontelles2009}.

\subsection{Tipo de Pesquisa (quanto aos objetivos)}
A presente pesquisa é exploratória, tendo em vista que tem por finalidade investigar os principais problemas gerados pelas arquiteturas tradicionais das plataformas de ensino da \textit{Internet}, orientando assim as decisões do sistema proposto \cite{fontelles2009}.

\subsection{Tipo de Pesquisa (quanto aos procedimentos técnicos)}
A pesquisa é classificada como bibliográfica, visto que possui fundamentacão em material já publicado. Por meio a análise e síntese de infomações, a pesquisa bibliográfica utilizada será crucial para fornecer direcionamento para o desenvolvimento da aplicação \cite{fontelles2009}.

\subsection{Tipo de Pesquisa (quanto ao desenvolvimento no tempo)}
A pesquisa será prospectiva, visto que estima os efeitos e vantagens futuras do sistema \cite{fontelles2009}.

\subsection{Processo Metodológico}
Na Universidade de Brasília, o Trabalho de Conclusão de Curso (TCC) é dividido em duas fases: TCC 1 e TCC 2. Na primeira etapa, o objetivo é estabelecer uma base teória sólida, que servirá como base para uma proposta de solução, que por sua vez será desenvolvida na segunda fase do trabalho.

Entre as atividades desenvolvidas no período de TCC 1 pela equipe, estão:
\begin{itemize}
    \item Definição do tema e da metodologia;
    \item Realizar pesquisa bibliográfica;
    \item Elaborar a proposta de solução;
    \item Definição da arquitetura do sistema;
    \item Desenvolver simulação do software;
\end{itemize}

\subsection{Busca e seleção de artigos}
Os idealizadores do projeto tiveram uma jornada particular até encontrar os materiais necessários para construir toda a escrita do presente documento. Primeiramente, iniciaram a pesquisa utilizando o software Publish or Perish, que se trata de um software que recupera e analisa citações acadêmicas com base em uma variedade de fontes de dados particular \cite{harzing2025publish}.

Após algumas tentativas utilizando várias palavras-chave e suas combinações no publish perish como filtros, os resultados alcançados não atenderam às expectativas iniciais, e não foi encontrado nenhum material de grande valia para a completude do trabalho em questão. A seguir, alguns dos termos utilizados pelos integrantes: 

\begin{itemize}
    \item Objetivos da descentralização p2p
    \item Sistemas computacionais p2p
    \item Importancia da descentralização computacional
    \item P2P network
    \item Computing decentralized system
    \item Decentralized computing
    \item Ipfs storage
    \item Decentralized web ipfs
    \item MVC architecture web
    \item Client server
    \item Software architecture
\end{itemize}

Diante das dificuldades encontradas, a equipe ampliou suas pesquisas em outras plataformas, tendo o Google como principal fonte de busca, e então obtiveram resultados mais satisfatórios. Para assegurar a qualidade e relevância dos materiais encontrados, foram selecionados os conteúdos publicados em fontes confiáveis e reconhecida autoridade acadêmica.

\section{Metodologia de desenvolvimento}
\subsection{Metodologias ágeis}
o metodologias ágeis de software surgiram a partir do manifesto ágil, que foi criado por 17 desenvolvedores no ano de 2001. Esse documento foi um marco para a indústria de software pois mudou a forma como era enxergado a gestão, desenvolvimento e entrega de projetos tecnológicos. 
Um conjunto de valores e princípios que priorizam a entrega contínua para o cliente, flexibilidade e colaboração entre a equipe formam o alicerce desse manifesto. Ele foi uma alternativa a forma que software era entendido até então por meio dos modelos tradicionais como o modelo cascata (Waterfall), que não raramente ocasionava mudanças inesperadas nas entregas e longos ciclos de desenvolvimento do projeto. Esse manifesto foi criado a partir da experiência desses desenvolvedores que estabeleceram quatro valores fundamentais: 

\begin{itemize}
    \item Indivíduos e interações mais que processos e ferramentas.
    \item Software em funcionamento mais que documentação abrangente.
    \item Colaboração com o cliente mais que negociação de contratos.
    \item Responder a mudanças mais que seguir um plano.
\end{itemize}

Apesar dos itens a direita na frase terem sua relevância no processo, o foco deve ser o que se vê a esquerda afim de garantir um desenvolvimento mais dinâmico e eficiente.

Além desses valores, o manifesto apresenta 12 princípios que guiam o desenvolvimento ágil: Entre eles destaca-se para esse projeto:

\begin{itemize}
    \item Construa projetos em torno de indivíduos motivados. Dê a eles o ambiente e o suporte necessário e confie neles para fazer o trabalho..
    \item O método mais eficiente e eficaz de transmitir informações para e entre uma equipe de desenvolvimento é através de conversa face a face.
    \item Contínua atenção à excelência técnica e bom design aumenta a agilidade.
    \item As melhores arquiteturas, requisitos e designs emergem de equipes auto-organizáveis.
    \item Entregar frequentemente software funcionando, de poucas semanas a poucos meses, com preferência à menor escala de tempo.
\end{itemize}

Esses princípios foram o base para as metodologias utilizadas no mercado. Algumas delas estão sendo utilizadas nesse TCC e serão abordadas posteriormente nesse capítulo.

\subsubsection{Extreme Programming (XP)}
Extreme Programming é uma metodologia ágil que valoriza os princípios de comunicação, simplicidade, feedback, respeito e coragem, com o objetivo de melhorar e desenvolver software enquanto prioriza as pessoas envolvidas no projeto \cite{extremeprogramming}. Sua abordagem iterativa e incremental permite maior adaptabilidade às mudanças, garantindo que as entregas sejam rápidas e alinhadas com os requisitos do projeto. 

Diante de uma equipe enxuta composta por apenas dois desenvolvedores, foram necessárias adaptações na metodologia para que ela atendesse às necessidades específicas do projeto. Assim, algumas práticas foram enfatizadas, enquanto outras foram ajustadas ou parcialmente adotadas.

\subsubsection{Particularidades do XP no Projeto}
Com uma pequena equipe, algumas práticas do XP foram mais valorizadas e amplamente utilizadas pela dupla idealizadora do trabalho. Entre elas, destaca-se a programação por pares, onde periodicamente os desenvolvedores realizaram o trabalho em uma chamada online, realizando a revisão do trabalho de forma cruzada quando isto não ocorria. Dessa forma, tais práticas garantiram maior qualidade ao código produzido, reduzindo a probabilidade de erros críticos.

Além disso, outras práticas do XP foram adotadas, com a finalidade de garantir a eficiência durante o desenvolvimento:

\begin{itemize}
    \item \textbf{Refatoração contínua}: O código foi revisado e aprimorado durante o desenvolvimento, para melhorar a legibilidade, eficiência e manutenção do sistema.
    \item \textbf{Desenvolvimento incremental e iterativo}: O sistema foi construído com pequenas entregas, permitindo que as adaptações fossem realizadas de acordo com o surgimento de requisitos.
    \item \textbf{Comunicação contínua}: A equipe manteve um fluxo constante de comunicação por meio de reuniões periódicas e mensagens assíncronas, garantindo que ambos os membros estivessem alinhados quanto às prioridades e decisões técnicas. As principais plataformas utilizadas para realizar tal comunicação foram o WhatsApp (para comunicação assíncrona) e o Discord (para comunicação síncrona).
    \item \textbf{Simplicidade do design}: O foco foi manter a arquitetura e as implementações o mais simples possível, evitando complexidades desnecessárias, facilitando assim o processo de desenvolvimento.
\end{itemize}

\subsubsection{Kanban}
O Kanban é uma metodologia ágil que tem como foco a visualização do fluxo de trabalho. A ideal central deste método de gestão é otimizar a entrega de valor otimizando e limitando o trabalho em progresso (\textit{Work in Progress - WIP}) \cite{kanban2025}. Com isso, foi possível evitar a sobrecarga de trabalho da equipe, tornando o processo de desenvolvimento mais eficiente.

O Kanban, diferente de outras metodologias, permite grande flexibilidade ao adaptar o fluxo de trabalho existente sem ciclos de desenvolvimento fixos. Dentre as principais características da metodologia, a equipe de desenvolvimento fez bastante uso dos seguintes:

\begin{itemize}
    \item \textbf{Visualização do fluxo de trabalho}: Com os quadros visuais, a equipe tenha uma visão clara das tarefas pendentes, em andamento e concluídas.
    \item \textbf{Limitação do trabalho em progresso (WIP)}: Com uma quantidade de trabalho limitada, a equipe teve um grande foco diante das tarefas dispostas nos quadros.
    \item \textbf{Gestão contínua do fluxo}: A equipe adaptou constantemente o fluxo de trabalho com base na análise dos gargalos e otimização dos processos.
\end{itemize}

\subsubsection{Trello}
O Trello é uma ferramenta de gerenciamento de projetos que tem por finalidade organizar as tarefas em um quadro visual. Ele é amplamente utilizado de forma conjunta com o método Kanban, dada sua interface intuitiva para manipulação de cartões e monitoramento do fluxo de trabalho.

\subsubsection{Uso do Kanban com Trello}
O Trello facilita a utilização do método Kanban, possibilitando a visualização e acompanhamento das tarefas do projeto de software de forma virtual \cite{campos2023trello}. Isso permite os desenvolvedores a melhor compreender aquilo que precisa ser feito, tendo uma clara visão a cerca dos gargalos do trabalho. Para garantir uma visão clara do trabalho, a equipe estruturou o quadro Kanban do projeto com as seguintes colunas principais:
z
\begin{itemize}
    \item \textbf{Backlog}: Apresenta as tarefas planejadas ainda pendentes.
    \item \textbf{Em progresso}: Atividades que estão em execução.
    \item \textbf{Revisão}: Tarefas que devem apenas ser revisadas para serem concluídas.
    \item \textbf{Concluído}: Atividades finalizadas e aprovadas.
\end{itemize}


\section{DevOps}
\textit{DevOps} é uma abordagem de desenvolvimento e operações, capaz de unir pessoas, processos e tecnologias, principalmente através da integração contínua (CI) e entrega contínua (CD), para lidar com funções que anteriormante eram tratadas isoladamente, como desenvolvimento, operações de TI e engenharia de qualidade e segurança \cite{microsoftdevops}.

\subsection{CI e CD}
A Entrega contínua é uma prática que garante que o código gerado pelo processo de integração contínua esteja pronto ser disponibilizado nos ambientes de desenvolvimento e produção. Este recurso possibilita a eficáfia, facilidade e rapizes para a entrega de novos recursos e atualizações aos usuários finais \cite{gomes2023}.

% A implementação contínua é responsável por eliminar a necessidade de realizar esse processo manualmente. Após um ciclo bem sucedido de integração e entrega contínua, o artefato gerado pelo \textit{CI} é importado pelo \textit{CD}, que então identifica que existem alterações prontas para serem lançadas nos ambientes de desenvolvimento e produção, permitindo assim o uso independente de alterações de diferentes equipes, aumentando a produtividade e reduzindo os ciclos de lançamento \cite{gomes2023}.

Essas duas práticas são essenciais para o fluxo \textit{DevOps}, permitindo a produção mais rápida e frequente de versões do código, criando uma conexão entre desenvolvimento e produção.

\subsection{Plataforma como Serviço (\textit{PaaS})}
A computação em nuvem consiste na entrega de recursos de TI sob demanda por meio da Internet, eliminando assim a necessidade de indivíduos e empresas gerenciarem seus próprios recursos físicos, pagando apenas pelo uso \cite{awscloudcomputing}.

AWS é um exemplo de \textit{PaaS} que oferece suporte completo à abordagem DevOps, intregando escala, segurança, automações e um amplo ecossistema de parceiros para o sistema desenvolvido \cite{awsdevops}.

\subsection{GitHub Actions e CI/CD}
GitHub Actions é uma tecnologia que permite automatizar, personalizar e executar fluxos de trabalho de desenvolvimento diretamente no repositório GitHub para construir, testar e implantar o código automaticamente \cite{githubactions}. Com isso, será possível criar diversos processos de CI/CD, visando automatizar algumas tarefas, como testes e \textit{deploy}.

\subsection{Docker}
Docker é uma plataforma que possibilita a crição de aplicações em ambientes isolados, que contém os recursos necessários para a aplicação funcionar. Assim, o docker será uma ferramenta importante para garantir que o software funcione em diversos ambientens, incluindo o dos desenvolvedores, fornecendo benefícos portabilidade e isolamento dentro do projeto \cite{dockerdocs}.

\subsection{Práticas Utilizadas}
A seguir, serão apresentadas as principais conceitos ligados as atividades realizadas dentro das metodologias propostas pela equipe.

\begin{itemize}
    \item \textbf{Programação em Par}: Sessões de programação para aumentar a qualidade do código e driblar rapidamente barreiras durante o processo de construção do software.
    \item \textbf{Reuniões diárias}: Também chamadas de dailys, as reuniões diárias servem para discutir o progresso alcançado no dia anterior, além de identificar obstáculos para o trabalho do dia.
    \item \textbf{Kanban}: Ferramenta para gerenciar o fluxo de trabalho de forma visual, ajudando a identificar gargalos.
    \item \textbf{Testes funcionais}: Desenvolvidos para garantir a conformidade com as funcionalidades idealizadas do software.
    \item \textbf{Testes unitários}: Desenvolvidos para garantir o funcionamento individual das menores partes de um software, como por exemplo funções.
    \item \textbf{Integração e Entrega Contínua}: Integrar e testar o código frequentemente para identificar problemas o quanto antes.
\end{itemize}

\section{Requisitos}
Os requisitos de um sistema estabelecem o que se espera do sistema. As funcionalidades e limitações acordadas com o cliente. Segundo \cite{sommerville2011} os requisitos podem ser divididos entre requisitos funcionais e não funcionais. O primeiro descreve as funcionalidades que se espera do produto, e os não funcionais que delimitam restrições sobre como o sistema deve operar.

\subsection{Levantamento de requisitos}
A elicitação de requisitos para o desenvolvimento do presente trabalho foi conduzida por meio do Design Thinking. Tal método consiste em oferecer potenciais contribuições para a solução de problemas complexos que buscam identificar, compreender e solucionar, de modo criativo, problemas presentes em diferentes contextos. Com isso, é um método eficiente para disciplinas que carregam características de interdisciplinaridade, como é exemplo a ciência da informação, que possui uma transformação acelerada de ferramentas, tecnologias, mecanismos e suportes. \cite{apocalypse2022}.

Diferente de outras abordagens tradicionais de levantamento de requisitos, que tomam como base documentos formais ou especificações técnicas, o Design Thinking promove uma análise mais empática e iterativa. No contexto deste trabalho, onde a proposta inovadora é uma arquitetura descentralizada, e não funcionalidades nunca vistas antes, essa abordagem se mostrou essencial para compreender as soluções de usabilidade e experiência já consolidados no mercado de educação online. Assim, técnicas como análise de mercado e benchmarking foram utilizadas para satisfazer os requisitos do sistema, garantindo a satisfação dos usuários e um modelo arquitetural descentralizado.

\subsubsection{Fase de Inpiração}
Nesta etapa, foram elicitadas as plataformas de aprendizado online mais utilizadas, identificando as funcionalidades comuns entre as mesmas. Além disso, foi realizado um mapeamento dos pontos positivos e negativos das soluções centralizadas.

\subsubsection{Análise de Benchmarking}
O benchmarking é uma técnica que permite a comparação entre diferentes produtos ou serviços, identificando padrões de funcionalidades, boas práticas, possíveis limitações, pontos fracos e fortes e diferenciais.

Para garantir que o sistema desenvolvido esteja alinhado com as expectativas do mercado, foi realizado uma análise de Benchmarking de algumas das principais soluções educacionais atuais: Hotmart e Udemy. Ambas possuem recursos consolidados para proporcionar a criação, venda e consumo de conteúdos digitais. Dessa forma, foi possível compreender quais funcionalidades são indispensáveis e quais aspectos podem ser melhorados com a arquitetura descentralizada.

A tabela a seguir foi elaborada analisando os principais aspectos dessas plataformas, considerando as funcionalidades centrais.

% \begin{table}[h]
%     \centering
%     \caption{Comparação de funcionalidades entre Hotmart e Udemy}
%     \begin{tabular}{|p{5cm}|p{5cm}|p{5cm}|}
%         \hline
%         \textbf{Critério} & \textbf{Hotmart} & \textbf{Udemy} \\
%         \hline
%         \textbf{Criação de cursos} & Permite a criação e hospedagem de cursos online. & Oferece ferramentas para criação e publicação de cursos. \\
%         \hline
%         \textbf{Integração de Pagamentos} & Sistema de pagamento integrado (HotPay). & Sistema de pagamento próprio (Udemy Payments). \\
%         \hline
%         \textbf{Métodos de Monetização} & Venda direta, assinaturas, afiliados. & Venda direta, cupons de desconto, promoções \\
%         \hline
%         \textbf{Afiliados} & Programa de afiliados robusto. & Programa de afiliados disponível. \\
%         \hline
%         \textbf{Personalização} & Personalização limitada da página de vendas. & Personalização do conteúdo do curso. \\
%         \hline
%         \textbf{Certificados} & Personalização limitada da página de vendas. & Personalização do conteúdo do curso. \\
%         \hline
%         \textbf{Suporte a Vídeos} & Hospedagem de vídeos integrada. & Hospedagem de vídeos com suporte a legendas. \\
%         \hline
%         \textbf{Ferramentas de Marketing} & Ferramentas avançadas de marketing e automação. & Ferramentas básicas de marketing e promoção. \\
%         \hline
%         \textbf{Análise de Dados} & Relatórios detalhados de vendas e desempenho. & Análises de desempenho do curso e dos alunos. \\
%         \hline
%         \textbf{Comunidade} & Comunidade de produtores e afiliados. & Comunidade de instrutores e alunos. \\
%         \hline
%         \textbf{Suporte ao Instrutor} & Suporte por e-mail e chat. & Suporte por e-mail e centro de ajuda. \\
%         \hline
%         \textbf{Aplicativo Móvel} & Aplicativo para alunos e produtores. & Aplicativo móvel para alunos. \\
%         \hline
%         \textbf{Integrações} & Integração com ferramentas de e-mail e CRM. & Integração com ferramentas de terceiros via API. \\
%         \hline
%         \textbf{Segurança} & Proteção contra pirataria e fraudes. & Proteção básica contra pirataria. \\
%         \hline
%         \textbf{Gamificação} & Recursos limitados de gamificação. & Recursos de gamificação (badges, quizzes). \\
%         \hline
%         \textbf{Acessibilidade} & Recursos básicos de acessibilidade. & Legendas e transcrições para vídeos. \\
%         \hline
%         \textbf{Idiomas} & Suporte a múltiplos idiomas. & Suporte a múltiplos idiomas. \\
%         \hline
%         \textbf{Preços} & Taxas sobre vendas e planos de assinatura. & Taxas sobre vendas e compartilhamento de receita. \\
%         \hline
%     \end{tabular}
%     \label{tab:comparacao_hotmart_udemy}
%     \vspace{5mm}
%     {\footnotesize Fonte: Autores} 
% \end{table}


\subsubsection{Fase de Definição}


\subsection{Sessões de Brainstorming}

O brainstorming é uma das técnicas utilizadas durante a concepção de um software que consiste na geração de ideias pelos participantes sem preocução com julgamentos, não importando a validez ou o quão desurptiva esssa ideia possa ser. Segundo \cite{osborn1953}, precursor da técnica, esta é a chave para a eficácia da técnica. 
Essa abordagem permite que esse grupo aborde diferentes faces do problema, promovendo ideias criativas e soluções eficazes. Segundo \cite{miro2025}, alguns dos benéficios dessa técnica são:

\begin{itemize}
    \item \textbf{Encoraja a criatividade}: Como dito anteriormente, as sessões de brainstoorming incentivam o pensamento livre e criativo em um ambiente livre. As ideias de vários participantes juntas, podem gerar uma solução, diferente para o problema.
    \item \textbf{Incentiva a colaboração e trabalho em equipe}: Além da solução de problemas, ele permite que os membros entendam o processo criativo de cada um, auxiliando no entrosamento da equipe.
    \item \textbf{Introduz muitas ideias rapidamente}: O brainstoorming encoraja os membros apresentarem muitas ideias em um período de tempo curto. Todas as ideias postas são documentadas, e após algum crivo, pode resultar na solução perfeita.
\end{itemize}

Foi realizada 2 sessões de brainstoorming, a primeira presencialmente e a segunda no Discord. O princípio do encontro, foi marcado por um alinhamento a respeito do problema a ser resolvido. Durante as sessões, foi utilizado o uso de notas adesivas digitais na plataforma, e uso de papel e caneta para registro presencialmente. As notas foram agrupadas por proximidade dos temas. Os resultados obtidos através desse processo está exibido no apêndice desse trabalho. 

Diante dos resultados dessa técnica e dos levantamento de requisitos, foi possível visualizar com mais segurança quais seriam as funcinalidades principais para que a plataforma se adaquasse ao que se espera, diante do mercado.
