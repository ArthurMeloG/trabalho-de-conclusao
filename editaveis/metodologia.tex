\chapter[Metodologia]{Metodologia}

\section{Extreme Programming (XP)}
Extreme Programming é uma metodologia ágil que valoriza a comunicação, simplicidade, feedback, respeito e coragem, com o objetivo de melhorar e desenvolver software enquanto prioriza as pessoas envolvidas no projeto \cite{extremeprogramming}. Diante de uma equipe enxuta composta por apenas dois desenvolvedores, foram necessárias adaptações na metodologia para que ela atendesse às necessidades específicas do projeto.

\subsection{Particularidades do XP no Projeto}
Com uma pequena equipe, algumas práticas do XP foram mais valorizadas e amplamente utilizadas pela dupla idealizadora do trabalho. Entre elas, destaca-se a programação por pares, onde periodicamente os desenvolvedores realizaram o trabalho em uma chamada online, realizando a revisão do trabalho de forma cruzada quando isto não ocorria.


\section{Kanban}
O Kanban é uma metodologia ágil que tem como foco a visualização do fluxo de trabalho. A ideal central deste método de gestão é otimizar a entrega de valor otimizando e limitando o trabalho em progresso \cite{kanban2025}.

\subsection{Trello}
O Trello é uma ferramenta de gerenciamento de projetos que tem por finalidade organizar as tarefas em um quadro visual. Ele é amplamente utilizado de forma conjunta com o método Kanban, dada sua interface intuitiva para manipulação de cartões e monitoramento do fluxo de trabalho.

\subsection{Uso do Kanban com Trello}
O Trello facilita a utilização do método Kanban, possibilitando a visualização e acompanhamento das tarefas do projeto de software de forma virtual \cite{campos2023trello}. Isso permite os desenvolvedores a melhor compreender aquilo que precisa ser feito, tendo uma clara visão a cerca dos gargalos do trabalho.

\section{Busca e seleção de artigos}
Os idealizadores do projeto tiveram uma jornada particular até encontrar os materiais necessários para construir toda a escrita do presente documento. Primeiramente, iniciaram a pesquisa utilizando o software Publish or Perish, que se trata de um software que recupera e analisa citações acadêmicas com base em uma variedade de fontes de dados particular \cite{harzing2025publish}.

Após algumas tentativas utilizando várias palavras-chave e suas combinações no publish perish como filtros, os resultados alcançados não atenderam às expectativas iniciais, e não foi encontrado nenhum material de grande valia para a completude do trabalho em questão. A seguir, alguns dos termos utilizados pelos integrantes: 

\begin{itemize}
    \item Objetivos da descentralização p2p
    \item Sistemas computacionais p2p
    \item Importancia da descentralização computacional
    \item P2P network
    \item Computing decentralized system
    \item Decentralized computing
    \item Ipfs storage
    \item Decentralized web ipfs
    \item MVC architecture web
    \item Client server
    \item Software architecture
\end{itemize}

Diante das dificuldades encontradas, a equipe ampliou suas pesquisas em outras plataformas, tendo o Google como principal fonte de busca, e então obtiveram resultados mais satisfatórios. Para assegurar a qualidade e relevância dos materiais encontrados, foram selecionados os conteúdos publicados em fontes confiáveis e reconhecida autoridade acadêmica.