\chapter[Metodologia]{Metodologia}

\section{Extreme Programming (XP)}
Extreme Programming é uma metodologia ágil que valoriza a comunicação, simplicidade, feedback, respeito e coragem, com o objetivo de melhorar e desenvolver software enquanto prioriza as pessoas envolvidas no projeto (http://www.extremeprogramming.org/). Diante de uma equipe enxuta composta por apenas dois desenvolvedores, foram necessárias adaptações na metodologia para que ela atendesse às necessidades específicas do projeto.

\subsection{Particularidades do XP no Projeto}
Com uma pequena equipe, algumas práticas do XP foram mais valorizadas e amplamente utilizadas pela dupla idealizadora do trabalho. Entre elas, destaca-se a programação por pares, onde periodicamente os desenvolvedores realizaram o trabalho em uma chamada online, realizando a revisão do trabalho de forma cruzada quando isto não ocorria.


\section{Kanban}
O Kanban é uma metodologia ágil que tem como foco a visualização do fluxo de trabalho. A ideal central deste método de gestão é otimizar a entrega de valor otimizando e limitando o trabalho em progresso. (Ref)

\subsection{Trello}
O Trello é uma ferramenta de gerenciamento de projetos que tem por finalidade organizar as tarefas em um quadro visual. Ele é amplamente utilizado de forma conjunta com o método Kanban, dada sua interface intuitiva para manipulação de cartões e monitoramento do fluxo de trabalho.

\subsection{Uso do Kanban com Trello}
